\documentclass[12pt,a4paper]{article}


\usepackage[english]{babel}
\usepackage{newlfont}
\usepackage[latin1]{inputenc}
\usepackage{lmodern}
\usepackage[T1]{fontenc}
\usepackage{fancyhdr} %per impostare il documento
\usepackage{indentfirst} %per identazione 
\usepackage{graphicx} %per i grafici
\usepackage{newlfont} %per usare font particolari
\usepackage{amssymb}
\usepackage{amsmath}
\usepackage{latexsym}
\usepackage{amsthm}
\usepackage{hyperref}
\usepackage{natbib}
\usepackage{listings}
\usepackage{tcolorbox}
\usepackage{fixltx2e}

%%%%%%%%
\linespread{1.2}    %comando per impostare l'interlinea
%%%%%%%%

\oddsidemargin=30pt \evensidemargin=20pt %impostano i margini
\textwidth=450pt\oddsidemargin=0pt

%%%%%%%


\begin{document}

%%%%%%%% FRONTESPIZIO %%%%%%%%

\begin{titlepage}
\begin{center}
{{\Large{\textsc{Alma Mater Studiorum $\cdot$ Universit\`a di
Bologna}}}} \rule[0.1cm]{15.8cm}{0.1mm}
\rule[0.5cm]{15.8cm}{0.6mm}
{\small{\bf SCUOLA DI SCIENZE\\
Dipartimento di Scienze Biologiche, Geologiche e Ambientali - BiGeA }}
\end{center}
\vspace{6mm}
\begin{center}  
    {Corso di Laurea Magistrale in\\
    \bf Scienze e Gestione della Natura (LM-60)}
\end{center}
\vspace{15mm}
\begin{center}
{\LARGE{\bf Mixing distributions:}}\\  
\vspace{3mm}
{\LARGE{\bf the colorist R package applied }}\\
\vspace{3mm}
{\LARGE{\bf to community distribution}}\\
\vspace{3mm}
{\LARGE{\bf estimate}}\\
\end{center}
\vspace{30mm}
\par
\noindent
\begin{minipage}[t]{0.47\textwidth}
{\large{\bf Relatore:\\
Prof. Duccio Rocchini}}\\

{\large{\bf Correlatori:\\
Elisa Marchetto\\ Enrico Tordoni}}
\end{minipage}
\hfill
\begin{minipage}[t]{0.47\textwidth}\raggedleft
{\large{\bf Presentata da:
\\

Ludovico Chieffallo}\\
\vspace{3mm}
Matricola: 0000949523}
\end{minipage}
\vspace{20mm}
\begin{center}
{\large{\bf 18 Marzo 2022\\

Anno Accademico 2020-2021 }}
\end{center}
\end{titlepage}

%%%%%%%%%%%%%%%%%%%%%% Dedica %%%%%%%%%%%%%%%%%%%%%
\begin{titlepage}                      
%
\thispagestyle{empty}                   %elimina il numero della pagina
\topmargin=6.5cm                        %imposta il margina superiore a 6.5cm
\raggedleft                             %incolonna la scrittura a destra
\large                                  %aumenta la grandezza del carattere
                                       
\em                                     %(corsivo) 
Da scrivere in seguito\ldots                      %\ldots lascia tre puntini
\newpage                                
\clearpage{\pagestyle{empty}\cleardoublepage}%non numera l'ultima pagina sinistra
\end{titlepage}
%%%%%%%%%%%%%%%%%%%%%%%%%%%%%%%%%%%%%%%%%%%%%%%%%%%
\linespread{1.2} 
\newpage
\tableofcontents
\linespread{1.5} 
\newpage
\section{Abstract}
 xxxx\\
 
 
\vspace{1cm}
\textbf{Keywords}: x,x,x

\newpage
\section{Introduction}
\textit{"The most unique feature of Earth is the existence of life, and the most extraordinary feature of life is its diversity."} \citep{cardinale}.\\
Biodiversity is the variety of all forms of life on earth but the most important point is that it is fundamental for the balance of this planet.
Biodiversity is fundamental in many respects: conversion of solar energy, decomposition of organic matter, pollination, sustenance, raw materials useful for our economy, etc.
Biodiversity is regulated by a very delicate balance.
Many species, despite their resilience, are unable to adapt to changes in such a fast time frame.
The consequence of this is that biodiversity is on the decline; currently there are about 142,500 species within the IUCN red list with over 40,000 species threatened with extinction \citep{IUCN}. This number is obviously indicative and underestimated compared to the real numbers, in fact, the next goal of the International Union for Conservation of Nature is to reach 160,000 species to be included in the list.
To address this problem, one of the practical aspects adopted is monitoring, with the consequent objective of proposing a conservation system for the investigated species.
In the ecological field, researchers use species distribution models (SDMs) to study the relationships between known occurrences of species and the characteristics of the ecological and environmental landscape.
These models are useful not only for understanding where species are in this historical moment, but they are important for understanding where they were present in the past and for predicting where they might be in the future, allowing for an in-depth study of community dynamics.
Invasive species are highly regarded as they can have devastating effects on native species and the habitat itself.
With the SDMs it is possible to predict the movements of these species and consequently also predict the possible arrival of invasive species.
There are mainly 3 categories of SDM: mechanistic, correlative and hybrid \citep{dor12}. While mechanistic models generally incorporate physiological, morphological and behavioral data \citep{kea}, correlative and hybrid models use geographic occurrence data.
The correlative SDM category is in turn divided into 3 types: presence-absence, presence-background and presence-only methods.
For our project, the correlative category of presence-absence was used based on occurrences taken from the Global Biodiversity Information Facility (GBIF) and creating artificial absences (pseudo-absences).
In general, the SDMs provide output in 2 different formats: probability (logistic regression, GAM, enhanced regression trees, random forests) and suitability (eg GARP, ENFA, MaxEnt). \\ \textit{'Probabilistic results are influenced by an uneven proportion of species presences and absences. Regardless of the environmental conditions, most of the probability values of presence in the study area will be high if referred to widespread species; instead, they will be low if the target species is a microendemism.'} \citep{fav}
The consequence is that the probabilistic results defined for the species can neither be compared nor combined and on the other hand, the suitability lacks a standard unit of measurement and this makes them in turn unmatchable.
Here we introduce the concept of favourability; in fact, the models based on favourability produce comparable values regardless of the presence and absence value of the species distributions and therefore it is possible to compare and combine models for different species and for different environmental factors.
The favourability values (F, ranging from 0 to 1) can be obtained directly from the probability values using this formula:

\\ \[ F=\dfrac{\frac{P}{(1-P)}}{\frac{n_1}{n_0}+\frac{P}{(1-P)}}\] \\

Where \textit{P} is the probability of the presence of a species, \textit{n\textsubscript{1}} is the number of registered presences, and \textit{n\textsubscript{0}} is the number of absences.\\
If the method used to calculate P is logistic regression, the probability is defined according to the following equation:
\\ \[    F=\dfrac{e^y}{\dfrac{n_1}{n_0}+{e^y}}          \] \\
Where \textit{e} is the base of the Neperian logarithms and \textit{y} is the “logit function”, that is a linear combination of environmental variables.

%DA AGGIUNGERE LA PARTE DI COLORIST E CVD QUASI FINITA (DA RIFINIRE)

\newpage
\section{Aim}
dff

\newpage
\section{Methods}
\subsection{Study area}
The study area chosen for this project was Europe, covering an area of approximately 6,184,800 km2.
The surface taken into consideration includes both member states of the European Union and external states, however, despite the fact that for historical and cultural reasons there is a part of Russia considered within the European territory of almost 4,000,000 km2, it was chosen to exclude it from the analysis.
In this vast area we found various types of habitats, with a very high rate of biodiversity that makes this surface a natural laboratory for studying the various species and their distribution.

\subsection{Data and filters}
The data taken into consideration were extracted from the databases of the Global Biodiversity Information Facility (GBIF) \citep{gbif} between 18 October 2021 and 10 December 2021.
This data has been carefully filtered, eliminating many parameters taken into account in the initial database.
First of all, all the occurrence data with coordinates defined as "not available" were eliminated, then all the data were filtered on the basis of observation, keeping only the parameters defined as "human observation" and "observation" thus excluding all museum occurrences, private collections, zoos etc. that could have distorted our distribution model.
The data was then filtered by area of interest (Europe excluding European Russia) including a period of 40 years (from 1981 to 2021).
The year 1981 was carefully chosen as the starting year for our study, since most of the occurrences of interest fell from 1981 onwards for all the species considered.
Subsequently, all occurrences with an uncertainty in the coordinates greater than 100 meters were excluded, to avoid a too pronounced bias in our model.
Concerning databases with very numerous occurrences (in some cases over a million occurrences) this step has allowed us to remove many not very useful occurrences in our database, which has made the analysis work faster and more efficient.

\subsubsection{Worldclim data}

For this project, climate data was downloaded from WorldClim (available for free on the website https://www.worldclim.org/data/index.html).
Worldclim offers an efficient set of past and present data. It has recently implemented its service by adding the possibility to download data regarding future forecasts \citep{wc}, \citep{ey}. This data is often used for ecological modeling \citep{bcw}. 
The 19 bioclimatic variables were downloaded at a spatial resolution of 2.5 minutes (about 4.5 km at the equator). The same parameter was used for both past and future data.
The 19 variables include: mean annual temperature, mean diurnal interval, isotherm, temperature seasonality, maximum temperature of the hottest month, minimum temperature of the coldest month, annual temperature range, average temperature of the wettest quarter, average temperature of the wettest quarter. dry, mean temperature of the hottest district, mean temperature of the coldest district, annual rainfall, precipitation of the wettest month, precipitation of the driest month, rainfall seasonality, precipitation of the wettest district, precipitation of the driest district, precipitation of the district warmer, colder neighborhood rainfall 
%DO I ENTER THE COLLINEARITY DATA HERE?

\subsubsection{Species data}
To identify the species for this project we used the database of the International Union for Conservation of Nature's Red List of Threatened Species (IUCN red list).
The IUCN since 1964 has compiled these lists of all species more or less endangered and assesses their survival prospects. Today the IUCN red list represents the most complete inventory of the extinction risk of species globally \citep {IUCN}.

All the species investigated in this project have been carefully selected following some fundamental parameters.
The first parameter used to select the species was the population trend. We have only chosen species with a decreasing population trend. It must be pointed out that the term "decreasing" is not always synonymous with danger (for example Querus Robur has a very high number of occurrences but the population is decreasing), but it must drive to investigate the causes of this trend and the possible responses in order to preserve the species.
The distribution and number of occurrences are then taken into consideration. The distribution had to be consistent with the choice of the area designated in the project; therefore, all the selected species were present on the European territory. Subsequently, only species with a number of occurrences greater than 35,000 were chosen. A high number of occurrences allowed us to create a better model with better random sampling capacity and thus to decrease the bias \citep {kaplan} .

\bigskip
{\noindent \textbf{\textit{Melanitta Fusca}}} 
\\
The first species treated was \textit{Melanitta fusca (Linnaeus, 1758)}, it is a sea duck with a wide distribution both in central Europe, where it is present in several states, and in northern Europe where most of its occurrences are recorded.
A part of the occurrences is found in the territory of Asian Russia which, however, we have not considered in our project.
From surveys in 1992-1993, when the estimate of the wintering population of north-western Europe was estimated at around 933,000 individuals, in the period 2007-2009 there was a decline close to 60\% in the Baltic Sea bringing the count to around 373,000 individuals \citep{skov}.
Currently the number of natural individuals is estimated at around 141,000-268,000 specimens \citep{IUCN}.
\textit{Melanitta fusca} is currently classified as vulnerable in the IUCN red list \citep{IUCN}
Climate change poses potentially the greatest threat to the species today and in the future . The decrease in the duration of the spring snow cover has been linked to the decline in populations in North America, probably due to trophic maladjustment \citep{drever}.
Directly correlated with climate change, ocean acidification could lead to a decrease in shellfish that make up a large part of the \textit{Melanitta} diet \citep{stein}, \citep{carb}.
Another major threat is by-catches in fishing gear, which occur in particular in wintering areas \citep*{dagys}.
In general, therefore, this species is threatened by various factors: climate change, non-native species and diseases, pollution, hunting.\citep{IUCN}
It is therefore a kind of interest and Obtaining trend estimates is a top priority \citep{IUCN}.\\
Data extraction: \citep{mela}

\bigskip
{\noindent \textbf{\textit{Miniopterus schreibersii}}} 
\\
\textit{Miniopterus schereibersii (Kuhl, 1817)} is a bat of the \textit{Miniotteridae} family.
It has a very large range that affects various states of central and eastern Europe.
The global population of \textit{Miniopterus schreibersii} is estimated to have decreased by at least 30\% across much of its range \citep{IUCN}. It is currently already extinct in Germany, Ukraine and Austria. In general, the habitat of this species is extremely fragmented. At present this species is classified as vulnerable \citep{IUCN}.
The threats to this species are manifold: pollution from agricultural waste, human intrusion for recreational or work purposes, hunting and trapping, and invasive prion-induced disease.
It is included in Annex II and IV of the EU Habitats and Species Directive and therefore requires special conservation measures, including the designation of special areas of conservation \citep{dir}.
Currently there are numerous projects financed by LIFE for this species, in Spain, Italy and Romania \citep{IUCN}.\\
Data extraction: \citep{minio}

\bigskip
{\noindent \textbf{\textit{Lagopus muta}}} 
\\
\textit{Lagopus muta (Montin, 1781)}, also known as ptarmigan, is a bird of the Phasianidae family; it is a stationary species that populates the arctic and subarctic areas of Eurasia and especially North America. For our analysis we took into account only the occurrences located in the European section (excluding European Russia).
Due to the size of its range it cannot be considered a vulnerable species, despite the population trend in constant decrease \citep{IUCN}.
For this reason we find this species cataloged as least concern in the IUCN red list \citep{IUCN}.
In Europe, population size is estimated to be decreasing at a rate approaching 30\% in 12.6 years (three generations) \citep{bird}.
The threats affecting this species are on a global scale; habitat degradation and overhunting have had negative effects on population trends \citep{mad}. Much habitat loss occurs due to the development of tourist facilities and collision with cables around ski resorts can cause mortality \citep{IUCN}.
Human presence results in a displacement of the species from their wintering habitat. Overgrazing by reindeer is thought to be causing a decline in Sweden \citep{sto}.
Another fundamental problem is climate change.
A decrease recorded in the Swiss Alps of around 30\% of the population over 10 years has been attributed to climate change \citep{de}.\\
Data extraction: \citep{pernice}

\bigskip
{\noindent \textbf{\textit{Quercus robur}}} 
\\
\textit{Quercus robur (Linnaeus, 1753)}, commonly known as oak, is a tree belonging to the Fagaceae family.
It has a very large range that covers a large part of central Europe.
Due to its vast range it is not possible to consider it as an endangered species even if the population is constantly decreasing. \citep{khe}. It is therefore considered "least to concern" in the IUCN red list. \citep{IUCN}
The population is potentially at risk of decline due to climate change, which could lead to increased disease risk, loss of suitable habitat and increased exposure to unsuitable weather conditions \citep{jon}, \citep{IUCN}.
In general, throughout Europe, the population is declining and this is due to urban and agricultural expansion for timber \citep{du}.
Furthermore, oak has historically been subjected to mortality events from external pathogens (such as Agrilus biguttatus \citep{eat}).
In the future it is expected that the range of this species will likely change causing a shift towards north and east \citep{ef}.
Global warming also increases the susceptibility to pathogen infections; this could lead to increased environmental stress from drought or floods, extreme temperatures and increased mortality \citep{jon}.\\
Data extraction:\citep{querc}


\subsection{SDM: species distribution model} 
The use of species distribution models (SDMs) to map and monitor the distribution of plants and animals has become increasingly important wehn aiming at understanding environmental change, population dynamics and ecological consequences \citep{mil}.
In addition to predicting the distribution of species, SDMs have become important for many studies that include the study of climate change, the identification of potential protected areas, the determination of places more or less susceptible to invasions, etc. \citep{mil}.

There are mechanistic models where the phenotypic traits of a given organism are translated into performance or fitness components using sets of equations and correlative models, which we have dealt with.
Correlative models can be of three types: presence-absence, presence-background, and presence-only methods.
The presence-absence model was used for this project.
The presence-absence methods intuitively use the observations of the occurrences and absences of species. The absences correspond to places where the species was not found during field sampling. When geo-referenced absences are not available, it is possible to create pseudo-absences (artificially created).
Presence-absence modeling methods include: logistic regression \citep{bri}, generalized linear models and generalized additive models \citep{guis}, random forest \citep{brei}, boosted regression trees \citep{Eli}, multivariate adaptive regression splines \\ \citep{mois}, and artificial neural networks \citep{tarr}.
The generalized linear model (GLM) was chosen for this project.
\subsubsection{SDM structure}

The development process of an SDM is mainly composed of 4 phases: preparation of the data necessary for the project, calculation of the models, evaluation of the models and finally application of the models. Doing the data preparation is the part that will take the longest \citep{sil}.
Once the objective of the project has been determined, it is necessary to define the area of the study.
The continent of Europe was taken into consideration for this project.
Species occurrence records are indispensable for calculating a correlative niche model. There are several sources of species presence records; for this project it was decided to use the GBIF-Global Biodiversity Information Facility online database \citep{yes}.
Once the presences were obtained, the pseudo-absences were created.
Pseudo-absences must be created with caution as their positioning can strongly influence the results of the models \citep{mass}.
The pseudo absences can be created in various ways, we have chosen the function (random points), this has allowed the creation of random points on the map excluding the points of presence.
A prevalence rate of 0.4 was set for the creation of the pseudo-absence points.
A sample of 5000 occurrences for each species was chosen, consequently the pseudo absences created were 12,500 according to this formula:\\ \[(P)0.4=\frac{5000(Pres)}{12.500 (Abs)}\] \\
The best number of pseudo absences depends on the modeling technique used\\ \citep{kan} \citep{mass} .
In the case of this project based on generalized linear models, a good number of pseudo-absences is represented by a value greater than 10,000.
At this point, having obtained these data, the key aspect was to check them to check for any errors.
The most common errors are represented by occurrences with wrong or non-existent coordinates.
For this reason, the data were filtered to obtain sampling uncertainty levels <100 meters and all records with unavailable coordinates were excluded.
Another common mistake can be the inversion of longitude and latitude;
very often to check the level of this error it is possible to simply map the points to see if they fall within the study area or not.
Consequently, the next step was to check for any points outside our map of interest.
Once the points were identified, they had to be removed from our analysis.
As for the environmental values, for our project the bioclimatic variables taken from WorldClim were chosen \citep{bcw}.Most modeling algorithms are sensitive to high levels of correlation between predictive variables \citep{dor13}.Therefore the correlation of one or more predictor variables can drastically change the quality of the model.High correlations can have two main consequences \citep{dor13}\citep{demn}:
1) the results will be over-adapted and 2) the response curves will not be independent, thus the curve response of a variable will not exclusively represent that variable, but will include interactions with other related variables.
Variables with very high correlation values (such as -0.7 / -0.8) should be excluded \citep{dor13}.
Another alternative used was that of PCA (principal component analysis) \citep{pere}.
The PCA summarizes the environmental variables in orthogonal factors, therefore completely unrelated, and these factors can be used to model the niche of the species. However, the interpretation of the model will be more difficult \citep{sil}. 
Other methodologies have also been tried to understand correlation; Parametric ( Pearson) or non-parametric ( Spearman) coefficients were tested \citep{fiel}.
The variance inflation factor (VIF) measured the correlation of each variable with a combination of all other variables in the model together \citep{dor13}. Therefore, VIF measures the multicollinearity or non-independence of the predictive variables. However, the parameters will continue to be fair, consistent and efficient.
The generalized linear model (GLM) was then calculated with the function (MULTIGLM).
The model obtained was then ported to \texttt{colorist } and following the workflow, the maps representing the distribution of these species and the level of specificity of each area within the European territory were obtained. 

\subsection{\texttt{colorist} R package}
The \texttt{colorist} package represents the beating heart of this project.
Graphical map representation plays a key role in research to determine where, how and why wildlife distribution changes over space and time.\citep{colo}
Understanding these mechanisms is one of the main objectives of ecology.\\\citep{and1}
Traditionally, biologists used simple range maps to describe the distribution of species and individuals in space and time \citep{brt} \\\citep{gri}, however nowadays specific distribution data has become incredibly more abundant and readily available.
Surely researchers can take advantage of a greater amount of high quality data, however this entails a series of challenges in visualizing this data that can often be confusing and unclear.\citep{colo}
Community science projects have collected hundreds of millions of observations of wildlife populations \citep{nat} \citep{sal} and technologies for monitoring individual animals have improved and diversified \citep{ks}. Data for populations and individuals is now shared through a dedicated data infrastructure \citep{gbif} \citep{kra} and we have increasingly detailed descriptions of where and when wildlife can be found.
This package emphasizes the use of color to indicate where, when and how consistently species can be found, to do this, \texttt{colorist} takes information from a stack of rasters, processes it and colors it in HCL (hue-chroma-luminance) in specific ways so that the occurrence, abundance, or density values have nearly equal perceptual weights in the resulting maps. \citep{colo}
This package can be used to represent obtained data, as in our case, but it can also be used to explore a dataset in order to know the distribution of the species concerned.
It is important to remember that the \texttt{colorist} functions have been developed to manage and visualize the data, but the interpretations are the responsibility of the researcher.\citep{colo}
The workflow is as follows: creation of the metrics that describe the distribution of the species concerned, creation of the colorimetric palette for representation, creation of the map based on metrics and palettes and finally a legend to be associated with the map.
In this project we used \texttt{colorist} to visualize the distribution of our species.
After extracting, filtering and analyzing the data, the generalized linear model (GLM) was chosen for the study on the species distribution, predictions were made from the obtained statistical model in order to obtain favourability, these data were imported into \texttt{colorist}.
The metrics\_pull function was used to transform the model values into intensity values ranging from 0 (0 probability of occurrence) to 1 (maximum stack occurrence value).
  In\texttt{colorist} we use the term "intensity" as a generic descriptor of normalized data values that can reflect the probability of relative occurrence, abundance, density or probability density.\citep{colo}
We can describe this first phase as preparative, in fact this extraction process preserves all the information in the original stack while preparing the layers for the next visualization.

\begin{lstlisting}

library (colorist)
metrics<-metrics_pull(fav4sp) 

\end{lstlisting}

\bigskip
The \textit{'fav4sp'} file is a stack containing the information of all the species considered.
Subsequently, having obtained the metrics, the function \textit{palette\_set} was used.

\begin{lstlisting}

palette<-palette_set(fav3sp)
\end{lstlisting}

\bigskip
At this point, having obtained the metrics, having a palette that can be used, we moved on to creating the map.
For the map we used the function \textit{map\_multiples}.
This function returns a multiframe maps.

\begin{lstlisting}

maps <- map_multiples (metrics, palette, ncol = 2,
lambda_i = -5, labels = names (fav4sp))

\end{lstlisting}
\bigskip
In this function it is interesting to underline the \textit{'lambda'} factor that allows us to change the intensity of the display for a more correct interpretation of the data.\\
The map gave us an idea of the distribution of each species, however the \textit{metrics\_distill} function was used to view all the species on the same map to see if and how they shared common space.
\begin{lstlisting}

metricsdist<- metrics_distill(fav4sp)

\end{lstlisting}
\bigskip
In the end, the legend was created with the \textit{legend\_set} function; for this it is necessary to use only the palette.
\\
\begin{lstlisting}
legend <- legend_set(palette, group_labels = names(fav4sp))

\end{lstlisting}
\bigskip
At this point we can interpret the data through the intensity and specificity parameters, a higher specificity value corresponds to a more intense color in the representation, consequently a lower specificity rate corresponds to a lower intensity of the color.
When the specificity value is lower it means that the species share common space while when it is high it means that portion of space is not shared.

\subsection{Climate change scenarios}
The SDMs can be transferred in different spatial and temporal scenarios, the algorithms can apply the formula that determines the distribution of the species to a set of environmental variables that correspond to future projections.
Today we find ourselves in a world that changes faster than any other period in history.    
The specific data regarding climate change are analyzed in detail by the IPCC \citep{ipcc}.
The Intergovernmental Panel on Climate Change (IPCC) is the United Nations body for the evaluation of science related to climate change, the IPCC was created to provide policymakers with regular scientific assessments on climate change, its implications and potential future risks, as well as to propose adaptation and mitigation options\citep{ipcc}.
Climate change is a direct consequence of the impact that human beings have on the planet and this has serious consequences on all living beings that populate the Earth.
In this project a future scenario was analyzed using the bioclimatic variables coming from WorldClim \citep{cmi} from 2021 to 2040; they were combined with our data and the same previous models were run so that we could compare them.
Studying how the distribution of species varies with changing climatic conditions is fundamental to understand the dynamics between species that previously occupied different territories and that now and in the future may find themselves in common spaces they are not used to, this leads to loss of biodiversity, new population dynamics and new strategies for the conservation of these species.
WorldClim's future data has been structured thanks to the Coupled Model Intercomparison Project (CMIP6) which is a collaborative framework designed to improve knowledge of climate change \citep{CMIP6}.


\subsection{Model performance}

\newpage
\section{Results}
oijhf

\newpage
\section{Discussion}
kfhj

\newpage
\section{Conclusion}

yyegd

\newpage
\section{Future studies}
\newpage
\section{Acknowledgments}
hhfb




\newpage
\section{Bibliography}
\begin{thebibliography}{999}

\bibitem[ IUCN  2021 ]{IUCN} 
IUCN. 2021. The IUCN Red List of Threatened Species. Version 2021-3. https://www.iucnredlist.org. Accessed on [18/10/2021].

\bibitem[Kaplan et al. 2014]{kaplan} 
Kaplan, R.M., Chambers, D.A. and Glasgow, R.E. (2014), Big Data and Large Sample Size: A Cautionary Note on the Potential for Bias. Clinical And Translational Science, 7: 342-346. https://doi.org/10.1111/cts.12178
 
\bibitem[Skov et al. 2011]{skov}
(Skov, H; Heinänen, S.; Žydelis, R.; Bellebaum, J.; Bzoma, S.; Dagys, M.; Durinck, J.; Garthe, S.; Grishanov, G.; Hario, M.; Kieckbusch, J.J.; Kube, J.; Kuresoo, A.; Larsson, K.; Luigujoe, L.; Meissner, W.; Nehls, H.W.; Nilsson, L.; Petersen, I.K.; Roos, M.M.; Pihl, S.; Sonntag, N.; Stock, A.; Stipniece, A.; Wahl, J. 2011. Waterbird Populations and Pressures in the Baltic Sea. Nordic Council of Ministers, Copenhagen.)

\bibitem[Drever et al. 2011]{drever}
Drever, M.C., Clarck, R.G., Derksen, C., Slattery, S.M., Toose, P., Nudds, T.D. 2011. Population vulnerability to climate change linked to timing of breeding in boreal ducks. Global Change biology 18(2): 480-492.

\bibitem[Steinacher et al. 2009]{stein}
Steinacher M. , Joos, F., Frolicher, T.L., Plattner, G.K., Doney, S.C. 2009. Imminent ocean acidification in the Arctic projected with the NCAR global coupled carbon cycle-climate model. Biogeosciences 6(515-533).

\bibitem[Carboneras et al. 2020]{carb}
Carboneras, C., Kirwan, G.M.\& Sharpe, C.J. 2020. Velvet Scoter (Melanitta fusca), version 1.0. In: J. del Hoyo, A. Elliott, J. Sargatal, D. A. Christie, and E. de Juana (eds), Birds of the World, Cornell Lab of Ornithology, Ithaca, NY, USA.

\bibitem[Dagys \& Hearn. 2018]{dagys} 
Dagys, M. \& Hearn, R. (compilers). 2018. International Single Species Action Plan for the Conservation of the Velvet Scoter (Melanitta fusca) Western Siberia \& Northern Europe/NW Europe population.

\bibitem[European Environment Agency. 2013]{dir}
European Environment Agency. 2013. Species: Miniopterus schreibersii. Report under the Article 17 of the Habitats Directive. Available at: https://bd.eionet.europa.eu/article17/reports2012/species/summary/. (Accessed: 20.10.2021).

\bibitem[BirdLife International. 2015]{bird}
BirdLife International. 2015. European Red List of Birds. Office for Official Publications of the European Communities, Luxembourg.

\bibitem[Madge \& McGowan. 2002]{mad}
Madge, S.; McGowan, P. 2002. Pheasants, partridges and grouse: including buttonquails, sandgrouse and allies. Christopher Helm, London.

\bibitem[Storch et al. 2007]{sto}
Storch, I. 2007. Grouse: status survey and conservation action plan 2006-2010. IUCN and World Pheasant Association, Gland, Switzerland \& Cambridge, UK/Fordingbridge, UK.

\bibitem[De Juana et al. 2016]{de}
de Juana, E., Kirwan, G.M. and Garcia, E.F.J. 2016. Rock Ptarmigan (Lagopus muta). In: del Hoyo, J., Elliott, A., Sargatal, J., Christie, D.A. and de Juana, E. (eds), Handbook of the Birds of the World Alive, Lynx Edicions, Barcelona.

\bibitem[Khela et al. 2012]{khe}
Khela, S. 2012. Quercus robur. Available at: http://www.iucnredlist.org/details/63532/1. (Accessed: november 2021).

\bibitem[Jonsson. 2012]{jon}
Jonsson, L. 2012. Impacts of climate change on pedunculate oak (Quercus robur L.) and Phytophthora activity in north and central Europe. Department of Physical Geography and Ecosystem Science, Lund Univeristy.

\bibitem[Ducousso et al. 2003]{du}
Ducousso, A. and Bordacs, S. 2003. Pedunculate and sessile oaks.

\bibitem[Eaton et al. 2016]{eat}
Eaton, E., Caudallo, G., Oliveira, S and de Rigo, D. 2016. Quercus robur and Quercus petraea in Europe: distribution, habitat, usage and threats. European Atlas of Forest Tree Species, Publ. Off. EU.

\bibitem[EFDAC. 2015]{ef}
EFDAC- European Forest Data Centre. 2015. Species Distribution. Available at: http://forest.jrc.ec.europa.eu/download/data/species-distribution/. (Accessed: october 2021).

\bibitem[WorldClim CMIP6. 2016]{wc}
WorldClim future data available at: https://www.worldclim.org/data/cmip6/cmip6climate.html. (Accessed: october 2021)

\bibitem[ Eyring et al. 2016 ]{ey}
Eyring, V., Bony, S., Meehl, G. A., Senior, C. A., Stevens, B., Stouffer, R. J., and Taylor, K. E.: Overview of the Coupled Model Intercomparison Project Phase 6 (CMIP6) experimental design and organization, Geosci. Model Dev., 9, 1937–1958, https://doi.org/10.5194/gmd-9-1937-2016, 2016.

\bibitem[WorldClim. 2021]{bcw}
WorldClim bioclimatic Variables. Available at: https://www.worldclim.org/data/bioclim.html. (Accessed: october 2021)

\bibitem[Andrewartha \& Birch. 1954]{and1}
Andrewartha, H. G., \& Birch, L. C. (1954). The distribution and abundance of animals. Chicago, IL: University of Chicago Press.

\bibitem[Burt. 1943]{brt}
Burt, W. H. (1943). Territoriality and home range concepts as applied to mammals. Journal of Mammalogy, 24(3), 346– 352. https://doi-org.ezproxy.unibo.it/10.2307/1374834

\bibitem[Grinnel. 1904]{gri}
Grinnell, J. (1904). The origin and distribution of the chest-nut-backed chickadee. The Auk, 21(3), 364– 382. https://doi-org.ezproxy.unibo.it/10.2307/4070199

\bibitem[Kays et al. 2015]{ks}
Kays, R., Crofoot, M. C., Jetz, W., \& Wikelski, M. (2015). Terrestrial animal tracking as an eye on life and planet. Science, 348(6240), aaa2478. https://doi-org.ezproxy.unibo.it/10.1126/science.aaa2478

\bibitem[INaturalist. 2020]{nat}
iNaturalist. (2020). iNaturalist. Retrieved from https://www.inaturalist.org

\bibitem[Sullivan et al. 2014]{sal}
Sullivan, B. L., Aycrigg, J. L., Barry, J. H., Bonney, R. E., Bruns, N., Cooper, C. B., … Kelling, S. (2014). The eBird enterprise: An integrated approach to development and application of citizen science. Biological Conservation, 169, 31– 40. https://doi-org.ezproxy.unibo.it/10.1016/j.biocon.2013.11.003

\bibitem[GBIF. 2020]{gbif}
GBIF. (2020). Global biodiversity information facility. Retrieved from http://gbif.org

\bibitem[Kranstauber et al. 2011]{kra}
Kranstauber, B., Cameron, A., Weinzerl, R., Fountain, T., Tilak, S., Wikelski, M., \& Kays, R. (2011). The Movebank data model for animal tracking. Environmental Modelling & Software, 26(6), 834– 835.

\bibitem[Schuetz \& Strimas-Mackey \& Auer. 2020]{colo}
Schuetz, JG, Strimas-Mackey, M, Auer, T. colorist: An r package for colouring wildlife distributions in space–time. Methods Ecol Evol. 2020; 11: 1476– 1482. https://doi-org.ezproxy.unibo.it/10.1111/2041-210X.13477

\bibitem[Lagopus muta. GBIF. 2021]{pernice}
GBIF.org (26 October 2021) GBIF Occurrence Download  https://doi.org/10.15468/dl.334ay7

\bibitem[Miniopterus schreibersii. GBIF. 2021]{minio}
GBIF.org (29 November 2021) GBIF Occurrence Download  https://doi.org/10.15468/dl.b4jh6s

\bibitem[Quercus robur L.. GBIF. 2021]{querc}
GBIF.org (10 December 2021) GBIF Occurrence Download  https://doi.org/10.15468/dl.t3xsw7

\bibitem[Melanitta fusca. GBIF. 2021]{mela}
GBIF.org (08 December 2021) GBIF Occurrence Download  https://doi.org/10.15468/dl.xzb96a

\bibitem[Miller. 2010]{mil}
(Miller, J. (2010), Species distribution models. Compass Geography, 4: 490-509. Https://doi.org/10.1111/j.1749-8198.2010.00351.x)

\bibitem[Brito et al. 1999]{bri}
J.C. Brito, O.S. Crespo, e.g. Paulo
Modelling wildlife distributions: logistic multiple regression vs overlap analysis
Ecograph. (Cop.), 3 (1999), pp. 251-260

\bibitem[Guisan et al. 2002]{guis}
A. Guisan, T.C.J. Edwards, T. Hastie
Generalized linear and generalized additive models in studies of species distributions: setting the scene
Ecol. Modell. (2002), pp. 89-100

\bibitem[Breiman. 1999]{brei}
L. Breiman
Random forest, Mach. Learn., 45 (1999), pp. 1-35, 10.1023/A:1010933404324

\bibitem[Elith et al. 2008]{Eli}
J. Elith, J.R. Leathwick, T. Hastie
A working guide to boosted regression trees
J. Anim. Ecol., 77 (2008), pp. 802-813, 10.1111/j.1365-2656.2008.01390.x

\bibitem[Moisen \& Frescino. 2002]{mois}
G.G. Moisen, T.S. Frescino
Comparing five modelling techniques for predicting forest characteristics
Ecol. Modell., 2–3 (2002), pp. 209-225

\bibitem[Tarroso et al. 2012]{tarr}
P. Tarroso, S.B. Carvalho, J.C. Brito
Simapse - simulation maps for ecological niche modelling
Meth. Ecol. Evol., 3 (2012), pp. 787-791, 10.1111/j.2041-210X.2012.00210.x

\bibitem[Sillero et al. 2021]{sil}
Neftalí Sillero, Salvador Arenas-Castro, Urtzi Enriquez‐Urzelai, Cândida Gomes Vale, Diana Sousa-Guedes, Fernando Martínez-Freiría, Raimundo Real, A.Márcia Barbosa,
Want to model a species niche? A step-by-step guideline on correlative ecological niche modelling,
Ecological Modelling,Volume 456,2021,ISSN 0304-3800, https://doi.org/10.1016/j.ecolmodel.2021.109671.

\bibitem[Yesson et al. 2007]{yes}
C. Yesson, P.W. Brewer, T. Sutton, N. Caithness, J.S. Pahwa, M. Burgess, W.A. Gray, R.J. White, A.C. Jones, F.A. Bisby, A. Culham
How global is the global biodiversity information facility?
PLoS ONE, 11 (2007), p. e1124

\bibitem[Barbet-Massin et al. 2012]{mass}
M. Barbet-Massin, F. Jiguet, C.H. Albert, W. Thuiller
Selecting pseudo-absences for species distribution models: how, where and how many?
Meth. Ecol. Evol., 3 (2012), pp. 327-338, 10.1111/j.2041-210X.2011.00172.x

\bibitem[Kanagaraj et al. 2013]{kan}
R. Kanagaraj, T. Wiegand, A. Mohamed, S. Kramer-Schadt
Modelling species distributions to map the road towards carnivore conservation in the tropics

\bibitem[Dormann et al. 2013]{dor13}
C.F. Dormann, J. Elith, S. Bacher, C. Buchmann, G. Carl, G. Carré, J.R.G. Marquéz, B. Gruber, B. Lafourcade, P.J. Leitão, T. Münkemüller, C. McClean, P.E. Osborne, B. Reineking, B. Schröder, A.K. Skidmore, D. Zurell, S. Lautenbach
Collinearity: a review of methods to deal with it and a simulation study evaluating their performance
Ecograph. (Cop.), 36 (2013), pp. 27-46, 10.1111/j.1600-0587.2012.07348.x

\bibitem[De Marco \& Nobrega. 2018 ]{demn}
P. De Marco, C.C. Nobrega
Evaluating collinearity effects on species distribution models: an approach based on virtual species simulation
PLoS ONE, 13 (2018), 10.1371/journal.pone.0202403

\bibitem[Perez et al. 2018]{pere}
G. Pérez i de Lanuza, N. Sillero, M.Á. Carretero
Climate suggests environment-dependent selection on lizard colour morphs
J. Biogeogr., 45 (2018), pp. 2791-2802, 10.1111/jbi.13455

\bibitem[Field et al. 2012]{fiel}
A. Field, J. Miles, Z. Field, Discovering Statistics using R, Sage Publications (2012)

\bibitem[IPCC 2022]{ipcc}
https://www.ipcc.ch/ (Accessed: January 2022)

\bibitem[WorldClim CMIP6]{cmi}
https://www.worldclim.org/data/cmip6/cmip6climate.html

\bibitem[Coupled Model Intercomparison Project 2022]{CMIP6}
https://www.cmcc.it/it/data-services-and-products/data-list/cmip6-scenario-simulations

\bibitem[Cardinale et al. 2012]{cardinale}
Cardinale, B., Duffy, J., Gonzalez, A. et al. Biodiversity loss and its impact on humanity. Nature 486, 59–67 (2012). https://doi-org.ezproxy.unibo.it/10.1038/nature11148

\bibitem[Favourability function. 2018]{fav}
(https://biogeografia-uma.com/wp-content/uploads/2018/09/The\_favourability\_function.pdf)

\bibitem[Dormann et al. 2012]{dor12}
C.F. Dormann, S.J. Schymanski, J. Cabral, I. Chuine, C. Graham, F. Hartig, M. Kearney, X. Morin, C. Römermann, B. Schröder, A. Singer
Correlation and process in species distribution models: bridging a dichotomy
J. Biogeogr., 39 (2012), pp. 2119-2131, 10.1111/j.1365-2699.2011.02659.x

\bibitem[Kearney \& Porter. 2009]{kea}

M. Kearney, W.P. Porter
Mechanistic niche modelling: combining physiological and spatial data to predict species’ ranges
Ecol. Lett., 4 (2009), pp. 334-350



\end{thebibliography}







\end{document}
